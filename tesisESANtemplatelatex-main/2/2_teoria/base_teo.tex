%% Vaaraible independiente Deep Learning Visión por computadora


\subsection{Inteligencia Artificial}
La inteligencia artificial (IA) es un campo de la informática que se concentra en crear sistemas y programas capaces de realizar tareas similares a como lo haría un ser humano. Estos sistemas están diseñados para aprender de datos, reconocer patrones, tomar decisiones y adaptarse y mejorar con la experiencia. 

El termino Inteligencia Artificial fue dado a conocer en el año 1956 durante la Conferencia Dartmouth en Hanover, Nuevo Hampshire (Estados Unidos), donde se propuso la posibilidad de crear una máquina que pudiera pensar como un ser humano.  No obstante, esta a estado presente mucho antes desde de trabajos de investigación, hasta en el cine y novelas representando como ciencia ficción. 
Actualmente, se ha convertido en una de las tecnologías revolucionarias y de mayor interés de investigación.

\subsection{ Aprendizaje Automático  (Maching Learning)}
Es un subcampo de la inteligencia artificial que se centra en el desarrollo de algoritmos y modelos. Permitiendo a las computadoras aprender a partir de datos ya etiquetados y tomar decisiones según la información brindada, esto identificando patrones en los datos y utilizando estos para hacer predicciones o tomar decisiones.
Este tiende a aplicarse en los siguientes temas: detección de fraudes, predicciones, clasificación, sistemas de recomendación.
Este aprendizaje requiere datos de entrenamiento para que funcionen, ademas de mayormente trabajar con data estructurada.
\subsection{Aprendizaje Profundo (Deep Learning)}
Es un subcampo del aprendizaje automático que utiliza redes neuronales artificiales con múltiples capas (profundas) para modelar y resolver problemas complejos. Las redes neuronales profundas se caracterisan por la necesidad de recibir grandes conjuntos de datos estos pueden ser estructurados o no estructurado. Con la finalidad de aprender representaciones jerárquicas de los datos, lo que permite la automatización de la extracción de características.
Este tiende a aplicarse en los siguientes temas: Reconocimiento de voz y visión por computadora, procesamiento de lenguaje natural (NLP), Diagnóstico médico, conducción autónoma, entre otros.

\subsection{Computer Vision}
Se centra en entrenar a las máquinas para que interpreten las imágenes o videos de manera similar a como lo hacen los humanos. Esto implica la adquisición, el procesamiento, el análisis y la comprensión de imágenes y datos visuales para automatizar tareas que requieren la visión humana.
Algunas tecnologías y algoritmos en esta área son:
\newcommand{\CVone}{ Comparación estadística: Los algoritmos son capaces de  realizar comparaciones y análisis detallados de los objetos, más allá de ubicarlos en un plano.}

\newcommand{\CVtwo}{Detección de objetos: Los algoritmos son capaces de localizar y claisficar varios objetos en una imagen o en videos. Algunos algoritmos son YOLO (You Only Look Once), SSD (Single Shot MultiBox Detector), Faster R-CNN }

\newcommand{\CVthree}{ Análisis de Movimiento: Capacidad de seguir el movimiento y la dirección de un objetos o personas en una secuencia de video. Ejemplos: Optical Flow, Tracking algorithms (Kalman Filter, Mean-Shift, etc.). }

\begin{itemize}
	\item \CVone
	\item \CVtwo
	\item \CVthree
\end{itemize}
